\section{Intro Lab}
By visiting \link{https://github.com/nymann/IntroLab} you can see a gif demo of what
the game looks like at this stage in the \texttt{README.md} file.

\subsection{Implemented features}
\begin{enumerate}
        \item Added Enemy ship that moves randomly and shoots bullets
        \item Bullets will automatically be removed when outside the screen area.
        \item Naive weapon cooldown\: (you can only shoot if there's less than $n$ bullets on screen)
        \item Colored enemy ship, player ship and bullets.
\end{enumerate}

\subsection{Running via Make}
Just like we in this course strive towards abstraction and utilising various
techniques to manage our dependencies, I think it's important to abstract the build and
running process as much as possible, to avoid being tied to an IDE (or even a
specific build system (Maven in this case). In order to do this I utilise \texttt{GNU
Make}, so the end user can simply invoke \texttt{make run} to run the
application and \texttt{make test} to run the test suite.
The benefit of this is that if at a later point we want to use another build tool like Gradle or Ant,
then we only need to change the Make targets.
