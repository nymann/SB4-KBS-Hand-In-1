% Feedback: you should also describe how you access the components using the Lookup. 
\section{NetBeans Lab 2}
\link{https://github.com/nymann/NetBeansLab2}

\subsection{Theory}
\subsubsection{Registration}

\subsubsection{Access}
The access works the same as in NetBeans Lab 1, however to show that the dynamic
loading works at runtime I now constantly update post entity processing
services, instead of only getting them once during initialisation.
This comes at a performance hit, since update is called on every new frame.

\begin{minted}{diff}
- for (...: postEntityProcessingServices) {
+ for (...: getPostEntityProcessingServices()) {
\end{minted}

\subsubsection{Silent Update}
We have previously seen in NetBeansLabs 1 how we make Netbeans aware of our Core
module, we do the same thing for SilentUpdate:

\begin{minted}{mf}
Manifest-Version: 1.0
OpenIDE-Module-Localizing-Bundle: omitted/Bundle.properties
OpenIDE-Module-Install: omitted/silentupdate/UpdateActivator.class
OpenIDE-Module-Layer: omitted/autoupdate/silentupdate/resources/layer.xml
\end{minted}
Note \texttt{org/netbeans/modules/autoupdate} is shortened to omitted

Our UpdateActivator class overrides the restored method just like our Core
Installer does.
I made fields for the various parameters it takes:
\begin{minted}{java}
...
private static final int startupDelay = 0;
private static final int timeBetweenRun = 2;
...
@Override
public void restored() {
  executorService.scheduleAtFixedRate(
    UpdateHandler::checkAndHandleUpdates,
    startupDelay,
    timeBetweenRun,
    TimeUnit.SECONDS
  );
}
...
\end{minted}

Here the executorService is a threadPoolService with a core pool size of 1.
So in the end what we get is a thread that's running on startup (delay 0) and
then waits 2 seconds between each successful execution of the check and handle
updates static method.
That method is responsible for handling updates, here updates is refering to
newly installed modules, removed modules and updates to existing modules. The
internal implementation relies on the netebans auto update api and specifically
the UpdateUnit class \cite{netbeans-update-unit}.

\subsection{Running the game}
\begin{minted}{sh}
# The path to the updates.xml path in Bundle.properties is hard-coded.
# You can fix it manually or:
make fix-bundle-path
# Via make target
make run

# Via maven
mvn install -DskipTests -f Asteroids/pom.xml
mvn nbm:run-platform -f Asteroids/application/pom.xml
\end{minted}

\subsection{Asteroids draw bug fix}
\begin{minted}{diff}
- float x = this.getShapeX()[0];
- float y = this.getShapeY()[0];
- float radius = this.getRadius();
- sr.addCircle(x, y, radius);
+ PositionPart pos = this.getPart(PositionPart.class);
+ sr.addCircle(pos.getX(), pos.getY(), this.getRadius());
\end{minted}
Until this lab the collision detection didn't look quite right for asteroids.
The reason being that Asteroids are being draw using the \texttt{addCircle}
method, which expect ($x$, $y$) to be the center of the Asteroid, and not the top left.

\subsection{Demo}
The dynamic load-unload demo can be seen on
\link{https://github.com/nymann/NetBeansLab2} or downloaded via:
\link{https://github.com/nymann/NetBeansLab2/raw/master/demo/dynamic-load-unload.mp4}

