\section{Java Lab}
\link{https://github.com/nymann/JavaLab}

See \ref{design-lab} for an analysis of dependencies and possible design
changes.

\subsection{Running the game}
\begin{minted}{sh}
# Via make target
make run

# Via maven
mvn install -f AsteroidsEntityFramework/pom.xml
mvn package -f AsteroidsEntityFramework/Core/pom.xml
java -jar bin/JavaLab.jar
\end{minted}

\subsection{Uber jar} \label{java-lab:uberjar}
I encountered a problem in this lab with building a "Uber jar" like I have done
in the other labs, this was due to the newly introduced \texttt{META-INF}
service files in this lab. In the effort of bundling everything together, the
file names conflict for each service provider implementation, resulting in only
one provider implementation for each type. If you were to unzip the jar file
manually, and append an entry to the \texttt{IPostEntityProcessingServicel} file,
everything would work again.

The solution I ended up with to solve this automatically was to switch to
\texttt{maven-shade-plugin} instead of the previously used
\texttt{maven-assembly-plugin}.

In the shade plugin, it is solved by adding a service resource transformer:
\inputminted{xml}{code/maven-shade.xml}

